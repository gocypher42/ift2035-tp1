\documentclass[12pt, letterpaper]{article}
\usepackage[utf8]{inputenc}

\setlength{\parskip}{0.5em}

\title{IFT 2035 \\ Travail pratique 1 - Interpréteur Slip \\ Rapport }
\author{ Marie-Anne Prud'Homme-Maurice (1054064) 
\\ Olivier Guénette (20154866)}
\date{21 octobre 2021}

\begin{document}
\maketitle

\section*{Une sorte de Lisp}

Dans le cadre du cours IFT 2035. Il nous a été demandé de concevoir une sorte
d'interpréteur Lisp en utilisant le language fonctionnel Haskell. Le travail
a pour but d'implanter une fonction qui finalise l'analyse syntaxique de
l'expression fournie ainsi que de la fonction eval qui permet d'évaluer
celle-ci.

Ce rapport décrit notre processus d'analyse, les problèmes rencontrés,
les décisions prises et notre expérience durant la création de
cet interpréteur.

\section*{Analyse et compréhension de l'énnoncé}

Comme dans tout travail la première étape consistait à comprendre la tâche à
réaliser.  Sans le cacher, Haskell et Lisp sont des nouveaux langages pour nous.
Juste ce fait rend la tâche du projet plus complexe.

À la suite de plusieurs lectures, nous avons commencé à reconnaitre des 
similitudes entre la structure de Slip et Haskell.

\section*{Problèmes rencontrés}

\subsection*{Problèmes de compréhension}

Dans le fichier fourni "exemples.slip", certains exemples contenaient la 
syntaxe de Lisp donc nous pensions devoir aussi la gérer en plus de celle 
du Lisp. Après consultation avec un auxiliaire d'enseignement, nous avons 
appris que ce n'était pas le cas et nous devions transformer ces 
expressions en syntaxe Lisp. Par exemple, l'expression (\textgreater 3 4) 
devient (4 (3 \textgreater)).

\subsection*{Élimination dynamique du sucre syntaxique}

Le premier problème que nous avons rencontré était le manque de dynamisme
dans l'analyse des Scons. En effet, nous avions mis directement le nombre 
de Scons attendu dans le pattern matching du s2l, ce qui créait des 
problèmes pour les exemples qui demandaient un nombre différent. Nous 
avons réglé ce problème en plaçant ces Scons dans une variable qui est
ensuite décomposée par s2l. 

\subsection*{Évaluation des Lfn}

L'évaluation des Lfn nous a causé problème dû à notre manque de connaissances 
du curring de haskell. Initialement, nous n'étions pas capables de créer une 
variable du type (Env -\textgreater Value -\textgreater Value). Ce qui nous à 
pousser à créer des fonctions auxiliaires inutiles dans le programme. 
L'évaluation des Lfn est quelque chose que nous avons résolu vers la fin du
travail pratique.

\section*{Solutions rejetées et choisies}

\subsection*{Implentation initial de eval slet et dlet}

Lorsque nous étions rendus à l'évaluation des slet et des dlet dans la fonction
eval, la première solution trouvée était de décortiquer l'information du let
en plusieurs parties. Nous avions donc implanté des fonctions auxiliaires
permettant de trouver toutes les variables définies dans le let ainsi que
l'expression finale à évaluer.

Cette méthode fonctionnait particulièrement bien pour les expressions n'incluant
pas de récursion, car cette solution remplaçait les variables du let par leur
valeur. Ainsi, si une expression fessait référence à une variable du let, le
système ne s'en souvenait pas.

Clairement cette méthode ne permettait pas l'évaluation de toutes les
expressions de slet et de dlet. En corrigeant ce problème, nous nous somme
rendu compte que la source du problème venait en effet du manque d'évaluation
en Vfn des Lfn. À la suite d’une bonne évaluation des Lfn en Vfn, cela nous a 
permis d'ajouter les variables du let dans leurs environnements respectifs pour 
ensuite évaluer l'expression finale.

\subsection*{Implentation initial de eval pour Lfn}

Comme mentionné plus haut, nous avions on problème avec l'évaluation des Lfn. 
La solution initiale était de décomposer le Lfn pour en extraire les variables 
inconnues ainsi que le corps de la fonction.

Cette méthode était pratique, car elle nous permettait de savoir exactement le
nombre de variables inconnues. Ainsi, nous pouvions valider si ces variables 
existaient avant même d'évaluer la fonction. En contrepartie, cette méthode 
ne sauvegardait en aucun cas les fonctions dans l'environnement, car nous ne 
les transformions pas en Value. Lors de l'implantation de la fonction fact,
il nous a fallu trouver une solution.

Nous avons trouvé la solution lorsque nous avons compris qu'une fonction lambda
pouvait accepter plus qu'un argument à la fois. Nous nous étions basés sur les 
fonctions pour créer l'environnement initial des équations standard.
Ainsi, nous avons pu convertire les Lexp de Lfn en Value Vfn et les insérer 
dans les environnements.

\pagebreak

\section*{Conclusion}

In fine, ce travail pratique nous a permis d'approfondir nos connaissances 
des langages fonctionnels. Certes, nous avons rencontré des problèmes. 
Cependant, ceux-ci étaient causés principalement par notre expérience très brève 
avec ces langages de programmation.  

\end{document}
