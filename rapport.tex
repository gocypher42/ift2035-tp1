\documentclass[12pt, letterpaper]{article}
\usepackage[utf8]{inputenc}

% \setlength{\parskip}{1em}

\title{IFT 2035 \\ Travail pratique 1 - Interpréteur Slip \\ Rapport }
\author{ Marie-Anne Prud'Homme-Maurice (Ajout Matricule) 
\\ Olivier Guénette (20154866)}
\date{21 octobre 2021}

\begin{document}
\maketitle

\section*{Une sorte de Lisp}

Dans le cadre du cours IFT 2035. Il nous a été demandé de concevoir une sorte
d'interpréteur Lisp en utilisant le language fonctionnel Haskell. Le travail
a pour buts d'implanter une fonction qui finalise l'analyse syntaxique de
l'expression fournie ainsi que de la fonction eval qui permet d'évaluer
celle-ci.
\\
\\
Ce rapport décri notre pocessus d'analyse, les problèmes rencontrés,
les décisions prisent et notre exérience durant la création de
cette interpréteur.

\section*{Analyse et compréhension de l'énnoncé}

Comme dans tout travail la première étape consistait à comprendre la tâche à
réaliser.  Sans le cacher, Haskell et Lisp sont des nouveaux language pour nous.
Juste ce fait rend la tâche du projet plus complexe.
\\
\\
Suite à plusieurs lectures, nous avons commencé à reconnaitre des similitude
entre la strucure de Slip et Haskell.

\section*{Problèmes rencontrés}

\subsection*{Problèmes de compréhension}

À faire (Marie)

\subsection*{Élimination dynamique du sucre syntaxique}
Le premier problèmes que nous avons rencontrés était le manque de dynamisme
dans l'analyse des Scons.

À faire (Marie)

\subsection*{Évaluation des Lfn}
L'évaluation des Lfn est quelque chose que nous avons résolue vers la fin du
travail pratique.

À Faire (Olivier)

\section*{Solutions rejetées et choisies}

\subsection*{Implentation initial de eval slet et dlet}

Lorsque nous étions rendu à l'évaluation des slet et des dlet dans la fonction
eval, la première solution trouvée était de décortiquer l'information du let
en plusieur parties. Nous avions donc implenté des fonctions auxilliares
permettant de trouver toutes les variables définient dans le let ainsi que
l'expression final à évaluer.
\\
\\
Cette méthode fonctionnait particulièrement bien pour les expressions n'incluant
pas de récursion, car cette solution remplacait les variables du let par leur
valeur. Ainsi, si une expression fesait référence à une variable du let, le
système ne s'en souvenait pas.
\\
\\
Clairement cette méthode ne permettait pas l'évaluation de toutes les
expressions de slet et de dlet. En corrigeant ce problèmes, nous nous somment
rendu compte que la source du problème fenait en effet du manque d'évaluation
en Vfn des Lfn. Suite à une bonne évaluation des Lfn en Vfn, cela nous a permi
d'ajouter les variables du let dans leur environnement respectif pour ensuite
évaluer l'expression finale.

\subsection*{Implentation initial de eval pour Lfn}

À faire (Olivier)

\section*{Conclusion}

À faire

\end{document}
