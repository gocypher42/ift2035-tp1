\documentclass[12pt, letterpaper]{article}
\usepackage[utf8]{inputenc}

\setlength{\parskip}{1em}

\title{IFT 2035 \\ Travail pratique 1 - Interpréteur Slip \\ Rapport }
\author{ Marie-Anne Prud'Homme-Maurice (Ajout Matricule) 
\\ Olivier Guénette (20154866)}
\date{21 octobre 2021}

\begin{document}
\maketitle

\section*{Une sorte de Lisp}

Dans le cadre du cours IFT 2035. Il nous a été demandé de concevoir une sorte 
d'interpréteur Lisp en utilisant le language fonctionnel Haskell. Le travail 
a pour buts d'implanter une fonction qui finalise l'analyse syntaxique de 
l'expression fournie ainsi que de la fonction eval qui permet d'évaluer 
celle-ci.

Ce rapport décris notre pocessus d'analyse, les problèmes rencontrés, 
les décisions prisent et notre exérience durant la création de 
cette interpréteur.

\section*{Analyse et compréhension de l'énnoncé}

Comme dans tout travail la première étape concisistait à comprendre la tâche à 
réaliser.  

\section*{Problèmes rencontrés}

\section*{Solutions rejetées et choisies}

\section*{Surprise}

Chose à parler:

Implantation de eval des Vfn

Solution inital des let

Compréhension de la structure des Scons

\section*{Conclusion}




\end{document}
 