\documentclass[12pt, letterpaper]{article}
\usepackage[utf8]{inputenc}

% \setlength{\parskip}{1em}

\title{IFT 2035 \\ Travail pratique 1 - Interpréteur Slip \\ Rapport }
\author{ Marie-Anne Prud'Homme-Maurice (Ajout Matricule) 
\\ Olivier Guénette (20154866)}
\date{21 octobre 2021}

\begin{document}
\maketitle

\section*{Une sorte de Lisp}

Dans le cadre du cours IFT 2035. Il nous a été demandé de concevoir une sorte
d'interpréteur Lisp en utilisant le language fonctionnel Haskell. Le travail
a pour buts d'implanter une fonction qui finalise l'analyse syntaxique de
l'expression fournie ainsi que de la fonction eval qui permet d'évaluer
celle-ci.
\\
\\
Ce rapport décri notre pocessus d'analyse, les problèmes rencontrés,
les décisions prisent et notre exérience durant la création de
cette interpréteur.

\section*{Analyse et compréhension de l'énnoncé}

Comme dans tout travail la première étape consistait à comprendre la tâche à
réaliser.  Sans le cacher, Haskell et Lisp sont des nouveaux language pour nous.
Juste ce fait rend la tâche du projet plus complexe.
\\
\\
Suite à plusieurs lectures, nous avons commencé à reconnaitre des similitude entre
la strucure de Slip et Haskell.


\section*{Problèmes rencontrés}

\subsection*{Élimination dynamique du sucre syntaxique}

\subsection*{Évaluation des Lfn}

\section*{Solutions rejetées et choisies}

\subsection*{Implentation initial de eval slet et dlet}

Lorsque nous étions rendu à l'évaluation des slet et des dlet dans la fonction
eval, la première solution trouvée était de décortiquer l'information du let
en plusieur parties. Nous avions donc implenté des fonctions auxilliares
permettant de trouver toutes les variables définient dans le let ainsi que
l'expression final à évaluer.

\subsection*{Implentation initial de eval pour Lfn}

\section*{Surprises}

\section*{Conclusion}

\end{document}
