\documentclass[12pt, letterpaper]{article}
\usepackage[utf8]{inputenc}

\setlength{\parskip}{1em}

\title{IFT 2035 \\ Rapport \\ Travail pratique 1 - Interpréteur Slip}
\author{Olivier Guénette (20154866) 
\\ Marie-Anne Prud'Homme-Maurice (Ajout Matricule)}
\date{21 octobre 2021}

\begin{document}
\maketitle

\section*{Une sorte de Lisp}

Dans le cadre du cours IFT 2035. Il a été demandé de concevoir une sorte 
d'interpréteur Lisp en utilisant le language fonctionnel Haskell. Le travail 
a pour buts d'implanter une fonction qui finalise l'analyse syntaxique de 
l'expression fournie ainsi que de la fonction eval qui permet 
d'évaluer \\ celle-ci.

Ce rapport décris notre pocessus d'analyse, les problèmes rencontrés 
et les décisions prisent durant la création de cette interpréteur.

\section{Analyse et compréhension de l'énnoncé}
Ceci sera la section que nous allons parler de notre analyse.

\section{Problèmes rencontrés}

\section{Solutions rejetées et choisies}

\section{Surprise}


\end{document}
 